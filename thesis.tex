\documentclass{my_thesis}
% MUST BE COMPILED IN PDF, NOT DVI


\def\defcolor{goldfish}         % Definition color
\def\theocolor{pond}            % Theorem color
\def\excolor{pond}              % Example color
\def\generaltheocolor{beach}    % Color for corollaries, lemmas, propositions, etc

%% Tables
\def\tableheadcolorA{pond}
\def\tableheadcolorB{beach!80}
\def\shade{beach!70} % For alternating table colors
\renewcommand{\arraystretch}{1.1} % for more space between rows of tables

% For alternating colors in table columns
\newcolumntype{C}{>{\columncolor{\shade}}c} % center
\newcolumntype{L}{>{\columncolor{\shade}}l} % left

\usepackage{lipsum}

\title{The title}
\author{The author}
\date{The date}

\begin{document}
\maketitle

\chapter{A chapter}

\section{Theorem environments}

\begin{theorem}
{Description}
{refthis}
This is a theorem.
\end{theorem}
\begin{proof}
    This is the proof.
\end{proof}
Reference to \Cref{theorem:refthis}

\begin{lemma}
{Description}
{refthis}
This is a lemma.
\end{lemma}
Reference to \Cref{lemma:refthis}

\begin{proposition}
{Description}
{refthis}
This is a proposition.
\end{proposition}
Reference to \Cref{proposition:refthis}

\begin{definition}
{Description}
{refthis}
This is a definition.
\end{definition}
Reference to \Cref{definition:refthis}

\begin{corollary}
{Description}
{refthis}
This is a corollary.
\end{corollary}
Reference to \Cref{corollary:refthis}

\begin{example}
{Description}
{refthis}
This is a example.
\end{example}
Reference to \Cref{example:refthis}

\begin{remark}
{Description}
{refthis}
This is a remark.
\end{remark}
Reference to \Cref{remark:refthis}

\section{Tables}
Using tableheadcolorA
\[
    \rowcolors{2}{\shade}{white} % Alternating row colors, starting from row 2
    \begin{array}{llll}
    \rowcolor{\tableheadcolorA} % Header color
        1 & 2 & 3 & 4\\
        a & b & c & d\\
        a & b & c & d\\
        a & b & c & d\\
        a & b & c & d\\
        a & b & c & d\\
        a & b & c & d\\
        a & b & c & d\\
        a & b & c & d\\
        a & b & c & d
    \end{array}
\]

Using tableheadcolorB

\[
    \rowcolors{2}{\shade}{white} % Alternating row colors, starting from row 2
    \begin{array}{llll}
    \rowcolor{\tableheadcolorB} % Header color
        1 & 2 & 3 & 4\\
        a & b & c & d\\
        a & b & c & d\\
        a & b & c & d\\
        a & b & c & d\\
        a & b & c & d\\
        a & b & c & d\\
        a & b & c & d\\
        a & b & c & d\\
        a & b & c & d
    \end{array}
\]

\end{document}
